\documentclass{beamer}

% --------------- PACHETE UTILE ---------------
\usepackage[utf8]{inputenc}
\usepackage[T1]{fontenc}
\usepackage[romanian]{babel}
\usepackage{graphicx}
\usepackage{amsmath, amssymb}
\usepackage{booktabs}
\usepackage{hyperref}
\usepackage{fontawesome5} % Pentru \faGithub

% --------------- TEMA ȘI CULORI ---------------
\usetheme{Madrid} 

% --------------- INFORMAȚII DESPRE PREZENTARE ---------------
\title[ChessFlow]{ChessFlow: O Platformă Modernă de Șah Online}
\subtitle{Proiect de Echipă}
\author[Fugulin, Păcurariu, Pojoga]{Fugulin Victor \and Păcurariu Răzvan Mihai \and Pojoga Dragoș-Florin}
\institute[FMI - Info]
% \logo{\includegraphics[height=1cm]{logo_upb.png}} % Adaugă logo dacă ai

\date{\today}

% --------------- ÎNCEPUTUL DOCUMENTULUI ---------------
\begin{document}

% ----- SLIDE-UL DE TITLU -----
\begin{frame}
    \titlepage
\end{frame}

% ----- SLIDE-UL DE CUPRINS -----
\begin{frame}
    \frametitle{Cuprins}
    \tableofcontents % Se generează automat pe baza secțiunilor
\end{frame}

% --------------- SECȚIUNEA 1: IDEEA PROIECTULUI ---------------
\section{Ideea Proiectului}

\begin{frame}
    \frametitle{Conceptul ChessFlow}
    \begin{itemize}
        \item \textbf{Viziune:} Dezvoltarea unei platforme web moderne pentru jocul de șah online, inspirată de Lichess.
        \item \textbf{Obiective principale:}
        \begin{itemize}
            \item Interfață utilizator (UI) intuitivă și experiență de joc (UX) fluidă.
            \item Motor de șah (Engine) performant.
            \item Game Master (GM) pentru logica și arbitrarea jocului.
            \item Arhitectură scalabilă și modulară.
        \end{itemize}
        \item \textbf{Realizări cheie până acum:}
        \begin{itemize}
            \item Sistem de autentificare (Login/Register).
            \item Nucleul motorului de șah (generare mutări, evaluare, căutare Alpha-Beta).
            \item Baze pentru Game Master (management timp).
        \end{itemize}
    \end{itemize}
\end{frame}

% --------------- SECȚIUNEA 2: ARHITECTURA ---------------
\section{Arhitectura Sistemului}

\begin{frame}
    \frametitle{Arhitectura Generală ChessFlow}
    \begin{figure}
        \centering
        \includegraphics[height=0.6\textwidth, width=0.9\textwidth
                  ]{DiagramaChessFlow.png}
        \caption{Componentele principale și fluxul de date.}
    \end{figure}
\end{frame}

\begin {frame}
 \textbf{Componente principale:}
    \begin{itemize}
        \item \textbf{Frontend Web (UI):} Interfața vizuală cu utilizatorul.
        \item \textbf{Backend Web (API):} Gestiune utilizatori, matchmaking, API joc.
        \item \textbf{Game Master (GM):} Logica de joc, validări, comunicare cu Engine-ul.
        \item \textbf{Chess Engine:} Calcul mutări, analiză poziții (comunicare via UCI).
        \item \textbf{Bază de Date:} Stocare persistentă.
    \end{itemize}
\end {frame}

\begin{frame}
\begin{figure}
\centering
\includegraphics[width=0.8\textwidth, keepaspectratio
                  ]{UsecaseChessFlow.png}
        \caption{Use case Diagram.}
\end{figure}
\end{frame}

\begin{frame}
    \frametitle{Tehnologii Utilizate}
    \begin{columns}[T] % Aliniere la top
        \begin{column}{0.5\textwidth}
            \textbf{Web Frontend \& Backend:}
            \begin{itemize}
                \item Interfață utilizator: \textbf{React}
                \item API & Logică server: \textbf{ASP.NET Core (C\#)}
                \item Comunicare: API RESTful
            \end{itemize}
        \end{column}
        \begin{column}{0.5\textwidth}
            \textbf{Game Engine \& Game Master:}
            \begin{itemize}
                \item Limbaj: \textbf{Rust} (pentru performanță și siguranță)
                \item Protocol comunicare Engine: UCI
            \end{itemize}
        \end{column}
    \end{columns}
    \vfill
    \textbf{Alte Tehnologii și Unelte:}
    \begin{itemize}
        \item Bază de date: \textbf{SQLite}
        \item Versionare cod: Git, \textbf{GitHub}
    \end{itemize}
\end{frame}

\begin{frame}
    \begin{figure}
        \centering
        \includegraphics[height=0.6\textwidth, width=0.9\textwidth
                  ]{ChessFlowDB.png}
        \caption{Diagrama ER}
    \end{figure}
\end{frame}

% --------------- SECȚIUNEA 3: ASPECTE TEHNICE INTERESANTE ȘI/SAU DIFICILE ---------------
\section{Aspecte Tehnice Relevante}

\begin{frame}
    \frametitle{Provocări în Dezvoltarea Engine-ului (Rust)}
    \begin{itemize}
        \item \textbf{Reprezentarea Eficientă a Tablei (Bitboards):}
        \begin{itemize}
            \item Operații rapide pe biți pentru generarea și validarea mutărilor.
            \item Curba de învățare și complexitatea implementării corecte.
        \end{itemize}
        \item \textbf{Algoritmi de Căutare Optimizată (Minimax cu Alpha-Beta Pruning):}
        \begin{itemize}
            \item Reducerea drastică a spațiului de căutare în arborele de joc.
            \item Necesită o implementare atentă pentru a evita erorile subtile.
        \end{itemize}
        \item \textbf{Securizarea Comunicației cu JWT între ASP.NET și Rust:}
        \begin{itemize}
            \item Asigurarea unei validări consistente a token-urilor și a politicilor de autorizare între ecosistemul .NET (C\#) și cel Rust.
            \item Depanarea problemelor de generare/validare a token-urilor între cele două platforme cu biblioteci și implementări diferite.
        \end{itemize}
    \end{itemize}
\end{frame}

\begin{frame}
    \frametitle{Provocări: Integrare și Game Master (Rust & ASP.NET)}
    \begin{itemize}
        \item \textbf{Comunicarea Inter-Proces (Backend API $\leftrightarrow$ Game Master $\leftrightarrow$ Engine):}
        \begin{itemize}
            \item Asigurarea unei comunicări eficiente și robuste (ex: gRPC, mesagerie, TCP/IP pentru UCI).
            \item Gestionarea stării partidelor distribuite pe multiple componente.
        \end{itemize}
        \item \textbf{Logica Complexă a Game Master-ului:}
        \begin{itemize}
            \item Validarea tuturor regulilor de șah (rocadă, en-passant, remize etc.).
            \item Management precis al timpului și al condițiilor de final de joc.
        \end{itemize}
        \item \textbf{Scalabilitate și Performanță:}
        \begin{itemize}
            \item Proiectarea sistemului pentru a gestiona un număr mare de jocuri concurente.
        \end{itemize}
    \end{itemize}
\end{frame}

% --------------- SECȚIUNEA 4: CE MAI AVEȚI DE IMPLEMENTAT ---------------
\section{Planuri de Viitor}

\begin{frame}
    \frametitle{Direcții Principale de Dezvoltare}
    \begin{itemize}
        \item \textbf{Finalizarea Interfeței Utilizator (React):}
        \begin{itemize}
            \item Afișare tablă interactivă (drag & drop, evidențiere mutări).
            \item Istoric mutări, timer live, opțiuni de joc.
        \end{itemize}
        \item \textbf{Extinderea Funcționalităților de Joc (Backend \& GM):}
        \begin{itemize}
            \item Matchmaking (găsire oponenți, mod guest).
            \item Sistem de ranking (ELO).
            \item Suport pentru diferite formate de timp.
        \end{itemize}
        \item \textbf{Optimizarea și Îmbunătățirea Engine-ului (Rust):}
        \begin{itemize}
            \item Iterative deepening, opening books, endgame tablebases.
            \item Testare și rafinare continuă a funcției de evaluare.
        \end{itemize}
        \item \textbf{Infrastructură și Testare Riguroasă:}
        \begin{itemize}
            \item Implementare CI/CD pipelines.
            \item Teste unitare, de integrare și UI/UX.
            \item Documentație și panou de administrare.
        \end{itemize}
    \end{itemize}
\end{frame}

% ----- SLIDE-UL DE CONCLUZII (OPȚIONAL, DAR RECOMANDAT) -----
\begin{frame}
    \frametitle{Concluzii}
    \begin{itemize}
        \item Am pus bazele solide pentru ChessFlow, cu un motor de șah funcțional și un sistem de autentificare.
        \item Am ales tehnologii moderne (React, ASP.NET, Rust) pentru performanță și scalabilitate.
        \item Provocările tehnice abordate demonstrează complexitatea proiectului.
        \item Următorii pași se concentrează pe dezvoltarea UI, funcționalități de joc complete și robustețe.
    \end{itemize}
    \vfill
    \begin{center}
        \textit{Suntem entuziasmați să continuăm dezvoltarea ChessFlow!}
    \end{center}
\end{frame}

% ----- SLIDE-UL DE MULȚUMIRI / ÎNTREBĂRI-----
\begin{frame}
    \frametitle{Mulțumiri și Întrebări}
    \begin{center}
        \Huge Vă mulțumim pentru atenție!
        \vspace{1cm} \\
        Întrebări?
        \vspace{0.5cm} \\
    \end{center}
    \begin{center}
        \faGithub \hspace{0.2em} \texttt{https://github.com/Dragos-Florin-Pojoga/ChessFlow}
    \end{center}
\end{frame}

\end{document}